\documentclass[11pt,a4paper]{article}
\usepackage[utf8]{inputenc}
\usepackage{amsmath, amssymb}
\usepackage{geometry}
\usepackage{listings}
\usepackage{xcolor}
\usepackage{ctex}
\usepackage{algorithm}
\usepackage{algpseudocode}
\usepackage{graphicx}

% 设置页面边距
\geometry{top=2.5cm, bottom=2.5cm, left=2.5cm, right=2.5cm}

% 设置代码样式
\lstset{
  language=C++,
  basicstyle=\ttfamily\footnotesize,
  keywordstyle=\color{blue}\bfseries,
  commentstyle=\color{green!50!black},
  stringstyle=\color{red},
  numbers=left,
  numberstyle=\tiny,
  stepnumber=1,
  breaklines=true,
  frame=single,
  backgroundcolor=\color{gray!10}
}

\title{实验报告:GraphicsLibrary}
\author{董奕柳}
\date{\today}

\begin{document}

\maketitle

\section{项目概述}
这是一个简单的图形函数库,用于绘制基本的几何图形和文本显示,包含以下功能:
\begin{itemize}
    \item 绘制直线段
    \item 绘制圆弧
    \item 绘制椭圆弧
    \item 多边形区域填充
    \item 显示名字
\end{itemize}

该项目采用 C++ 编写,并使用 CMake 构建工具。

\section{功能设计}

\subsection{绘制直线段}
直线段的绘制基于 Bresenham 算法,其核心思想是通过整数计算近似实现直线段的绘制,避免使用浮点运算,从而提高效率。算法的数学公式及伪代码如下:

\subsubsection{数学公式}
假设起始点为 \((x_1, y_1)\),终点为 \((x_2, y_2)\),直线段的斜率为 \(k = \frac{\Delta y}{\Delta x}\),其中:
\[
\Delta x = x_2 - x_1, \quad \Delta y = y_2 - y_1
\]

Bresenham 算法通过计算误差项 \(e\) 来决定当前像素的位置更新:
\[
e = 2\Delta y - \Delta x
\]

根据误差更新规则:
- 若 \(e > 0\),则表示需要调整纵坐标 \(y\): 
\[
e = e - 2\Delta x
\]
- 否则,只调整横坐标 \(x\): 
\[
e = e + 2\Delta y
\]

\subsubsection{算法描述}
\begin{algorithm}[H]
\caption{Bresenham 绘制直线段算法}
\begin{algorithmic}[1]
\Require 起始点 \((x_1, y_1)\),终点 \((x_2, y_2)\)
\Ensure 绘制从 \((x_1, y_1)\) 到 \((x_2, y_2)\) 的直线段
\State 计算 \(\Delta x = |x_2 - x_1|\), \(\Delta y = |y_2 - y_1|\)
\State 初始化误差项 \(e = 2 \Delta y - \Delta x\)
\State 设置步长 \(y_{\text{step}} = 1\) 若 \(y_2 > y_1\),否则 \(y_{\text{step}} = -1\)
\State 初始化 \(y = y_1\)
\For{\(x = x_1\) to \(x_2\)}
    \State 绘制点 \((x, y)\)
    \If{\(e > 0\)}
        \State \(y \gets y + y_{\text{step}}\)
        \State \(e \gets e - 2 \Delta x\)
    \EndIf
    \State \(e \gets e + 2 \Delta y\)
\EndFor
\end{algorithmic}
\end{algorithm}

\subsection{绘制椭圆弧}
椭圆弧的绘制基于参数方程,其核心思想是使用角度增量逐点计算椭圆弧上的像素点。特别地,当椭圆的长轴和短轴相等时,椭圆弧就退化为圆弧。

\subsubsection{数学公式}
椭圆弧的参数方程为:
\[
(x, y) = (cx + a \cos \theta, cy + b \sin \theta), \quad \theta \in [\text{start}, \text{end}]
\]
其中:
- \((cx, cy)\) 为椭圆的中心坐标;
- \(a\) 为椭圆的长轴半径;
- \(b\) 为椭圆的短轴半径;
- \(\theta\) 为椭圆弧的角度。

当 \(a = b\) 时,椭圆弧退化为圆弧,其参数方程变为:
\[
(x, y) = (cx + r \cos \theta, cy + r \sin \theta), \quad \theta \in [\text{start}, \text{end}]
\]
其中 \(r = a = b\) 为圆的半径。

\subsubsection{算法描述}
\begin{algorithm}[H]
\caption{椭圆弧绘制算法}
\begin{algorithmic}[1]
\Require 椭圆中心 \((cx, cy)\),长轴 \(a\),短轴 \(b\),起始角度 \(\text{start}\),终止角度 \(\text{end}\)
\Ensure 绘制从起始角度到终止角度的椭圆弧
\State 设置步长 \(\text{step}\) 用于角度增量
\For{\(\theta = \text{start}\) to \(\text{end}\) step \(\text{step}\)}
    \State \(x \gets cx + a \cos \theta\)
    \State \(y \gets cy + b \sin \theta\)
    \State 绘制点 \((x, y)\)
\EndFor
\end{algorithmic}
\end{algorithm}

\subsection{多边形填充}
多边形填充基于扫描线算法,其核心思想是逐行扫描像素并填充多边形内部区域。

\subsubsection{数学公式}
设多边形的顶点集合为 \((v_1, v_2, \dots, v_n)\),对于扫描线 \(y = k\),交点的 \(x\)-坐标可以通过多边形的边方程计算:
\[
x = x_1 + \frac{(k - y_1)(x_2 - x_1)}{y_2 - y_1}, \quad y_1 \leq k < y_2
\]

\subsubsection{算法描述}
\begin{algorithm}[H]
\caption{扫描线多边形填充算法}
\begin{algorithmic}[1]
\Require 多边形顶点集合 \((v_1, v_2, \dots, v_n)\)
\Ensure 填充多边形内部
\State 计算多边形的最小 \(y\)-坐标 \(y_{\text{min}}\) 和最大 \(y\)-坐标 \(y_{\text{max}}\)
\For{\(y = y_{\text{min}}\) to \(y_{\text{max}}\)}
    \State 找到扫描线与多边形边的交点集合
    \State 按 \(x\)-坐标对交点排序
    \For{每对交点 \((x_{\text{left}}, x_{\text{right}})\)}
        \State 填充从 \(x_{\text{left}}\) 到 \(x_{\text{right}}\) 之间的像素
    \EndFor
\EndFor
\end{algorithmic}
\end{algorithm}

\section{结果展示}

\begin{figure}[htbp]
    \centering
    \includegraphics[width=0.8\textwidth]{graphic_result.png}
    \caption{通过图形库绘制的图形效果展示}
    \label{fig:result}
\end{figure}

图1展示了通过本图形库绘制的不同图形及其效果,包括以下几个部分:

\begin{itemize}
    \item \textbf{红色直线段}:从点 \((50, 50)\) 到点 \((400, 50)\),在窗口的顶部水平排列。
    \item \textbf{绿色半圆弧}:圆心位于 \((300, 200)\),半径为 100,起始角度为 0 度,终止角度为 180 度,形成一个半圆。
    \item \textbf{蓝色椭圆弧}:圆心位于 \((500, 200)\),长轴为 100,短轴为 60,起始角度为 0 度,终止角度为 180 度,形成一个椭圆弧。
    \item \textbf{黄色三角形}:顶点分别位于 \((150, 150)\)、\((250, 150)\) 和 \((200, 250)\),并进行了填充。
    \item \textbf{紫色文本}:在位置 \((150, 350)\) 绘制的“YILIU DONG”文本,展示了如何在图形界面中插入文本。
\end{itemize}

\section{总结}
通过本图形库,我们能够方便地绘制直线段、圆弧、椭圆弧和多边形填充,并实现文本显示功能。

\end{document}
