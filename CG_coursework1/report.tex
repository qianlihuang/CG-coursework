\documentclass[11pt,a4paper]{article}
\usepackage[utf8]{inputenc}
\usepackage{amsmath, amssymb}
\usepackage{geometry}
\usepackage{listings}
\usepackage{xcolor}
\usepackage{ctex}
\usepackage{algorithm}
\usepackage{algpseudocode}
\usepackage{graphicx}

% 设置页面边距
\geometry{top=2.5cm, bottom=2.5cm, left=2.5cm, right=2.5cm}

% 设置代码样式
\lstset{
  language=C++,
  basicstyle=\ttfamily\footnotesize,
  keywordstyle=\color{blue}\bfseries,
  commentstyle=\color{green!50!black},
  stringstyle=\color{red},
  numbers=left,
  numberstyle=\tiny,
  stepnumber=1,
  breaklines=true,
  frame=single,
  backgroundcolor=\color{gray!10}
}

\title{实验报告:GraphicsLibrary}
\author{董奕柳}
\date{\today}

\begin{document}

\maketitle

\section{项目概述}
这是一个简单的图形函数库,用于绘制基本的几何图形和文本显示,包含以下功能:
\begin{itemize}
    \item 绘制直线段
    \item 绘制圆弧
    \item 绘制椭圆弧
    \item 多边形区域填充
    \item 显示名字
\end{itemize}

该项目采用 C++ 编写,并使用 CMake 构建工具。

\section{功能设计}

\subsection{绘制直线段}
直线段的绘制基于 Bresenham 算法,其核心思想是通过整数计算近似实现直线段的绘制,避免使用浮点运算,从而提高效率。算法的数学公式及伪代码如下:

\subsubsection{数学公式}
假设起始点为 \((x_1, y_1)\),终点为 \((x_2, y_2)\),直线段的斜率为 \(k = \frac{\Delta y}{\Delta x}\),其中:
\[
\Delta x = x_2 - x_1, \quad \Delta y = y_2 - y_1
\]

Bresenham 算法通过计算误差项 \(e\) 来决定当前像素的位置更新:
\[
e = 2\Delta y - \Delta x
\]

根据误差更新规则:
- 若 \(e > 0\),则表示需要调整纵坐标 \(y\): 
\[
e = e - 2\Delta x
\]
- 否则,只调整横坐标 \(x\): 
\[
e = e + 2\Delta y
\]

\subsubsection{算法描述}
\begin{algorithm}[H]
\caption{Bresenham 绘制直线段算法}
\begin{algorithmic}[1]
\Require 起始点 \((x_1, y_1)\),终点 \((x_2, y_2)\)
\Ensure 绘制从 \((x_1, y_1)\) 到 \((x_2, y_2)\) 的直线段
\State 计算 \(\Delta x = |x_2 - x_1|\), \(\Delta y = |y_2 - y_1|\)
\State 初始化误差项 \(e = 2 \Delta y - \Delta x\)
\State 设置步长 \(y_{\text{step}} = 1\) 若 \(y_2 > y_1\),否则 \(y_{\text{step}} = -1\)
\State 初始化 \(y = y_1\)
\For{\(x = x_1\) to \(x_2\)}
    \State 绘制点 \((x, y)\)
    \If{\(e > 0\)}
        \State \(y \gets y + y_{\text{step}}\)
        \State \(e \gets e - 2 \Delta x\)
    \EndIf
    \State \(e \gets e + 2 \Delta y\)
\EndFor
\end{algorithmic}
\end{algorithm}

\subsection{绘制椭圆弧}
椭圆弧的绘制基于参数方程,其核心思想是使用角度增量逐点计算椭圆弧上的像素点。特别地,当椭圆的长轴和短轴相等时,椭圆弧就退化为圆弧。

\subsubsection{数学公式}
椭圆弧的参数方程为:
\[
(x, y) = (cx + a \cos \theta, cy + b \sin \theta), \quad \theta \in [\text{start}, \text{end}]
\]
其中:
- \((cx, cy)\) 为椭圆的中心坐标;
- \(a\) 为椭圆的长轴半径;
- \(b\) 为椭圆的短轴半径;
- \(\theta\) 为椭圆弧的角度。

当 \(a = b\) 时,椭圆弧退化为圆弧,其参数方程变为:
\[
(x, y) = (cx + r \cos \theta, cy + r \sin \theta), \quad \theta \in [\text{start}, \text{end}]
\]
其中 \(r = a = b\) 为圆的半径。

\subsubsection{算法描述}
\begin{algorithm}[H]
\caption{椭圆弧绘制算法}
\begin{algorithmic}[1]
\Require 椭圆中心 \((cx, cy)\),长轴 \(a\),短轴 \(b\),起始角度 \(\text{start}\),终止角度 \(\text{end}\)
\Ensure 绘制从起始角度到终止角度的椭圆弧
\State 设置步长 \(\text{step}\) 用于角度增量
\For{\(\theta = \text{start}\) to \(\text{end}\) step \(\text{step}\)}
    \State \(x \gets cx + a \cos \theta\)
    \State \(y \gets cy + b \sin \theta\)
    \State 绘制点 \((x, y)\)
\EndFor
\end{algorithmic}
\end{algorithm}

\subsection{多边形填充}
多边形填充基于扫描线算法,其核心思想是逐行扫描像素并填充多边形内部区域。

\subsubsection{数学公式}
设多边形的顶点集合为 \((v_1, v_2, \dots, v_n)\),对于扫描线 \(y = k\),交点的 \(x\)-坐标可以通过多边形的边方程计算:
\[
x = x_1 + \frac{(k - y_1)(x_2 - x_1)}{y_2 - y_1}, \quad y_1 \leq k < y_2
\]

\subsubsection{算法描述}
\begin{algorithm}[H]
\caption{扫描线多边形填充算法}
\begin{algorithmic}[1]
\Require 多边形顶点集合 \((v_1, v_2, \dots, v_n)\)
\Ensure 填充多边形内部
\State 计算多边形的最小 \(y\)-坐标 \(y_{\text{min}}\) 和最大 \(y\)-坐标 \(y_{\text{max}}\)
\For{\(y = y_{\text{min}}\) to \(y_{\text{max}}\)}
    \State 找到扫描线与多边形边的交点集合
    \State 按 \(x\)-坐标对交点排序
    \For{每对交点 \((x_{\text{left}}, x_{\text{right}})\)}
        \State 填充从 \(x_{\text{left}}\) 到 \(x_{\text{right}}\) 之间的像素
    \EndFor
\EndFor
\end{algorithmic}
\end{algorithm}

\section{结果展示}

\begin{figure}[htbp]
    \centering
    \includegraphics[width=0.8\textwidth]{graphic_result.png}
    \caption{通过图形库绘制的图形效果展示}
    \label{fig:result}
\end{figure}

图1展示了通过本图形库绘制的不同图形及其效果,包括以下几个部分:

\begin{itemize}
    \item \textbf{红色直线段}:从点 \((50, 50)\) 到点 \((400, 50)\),在窗口的顶部水平排列。
    \item \textbf{绿色半圆弧}:圆心位于 \((300, 200)\),半径为 100,起始角度为 0 度,终止角度为 180 度,形成一个半圆。
    \item \textbf{蓝色椭圆弧}:圆心位于 \((500, 200)\),长轴为 100,短轴为 60,起始角度为 0 度,终止角度为 180 度,形成一个椭圆弧。
    \item \textbf{黄色三角形}:顶点分别位于 \((150, 150)\)、\((250, 150)\) 和 \((200, 250)\),并进行了填充。
    \item \textbf{紫色文本}:在位置 \((150, 350)\) 绘制的“YILIU DONG”文本,展示了如何在图形界面中插入文本。
\end{itemize}

\section{总结}
通过本图形库,我们能够方便地绘制直线段、圆弧、椭圆弧和多边形填充,并实现文本显示功能。

\appendix
\section{附录}

\subsection{README.md}
\begin{lstlisting}[language=Markdown]
# GraphicsLibrary

## 项目概述
这是一个简单的图形函数库,有绘制直线段、圆弧、椭圆弧、填充多边形区域以及显示名字等功能。该项目采用 C++ 编写,并使用 CMake 构建工具。

## 功能列表
- 绘制直线段
- 绘制圆弧
- 绘制椭圆弧
- 多边形区域填充
- 显示名字

## 编译与运行

## 依赖

- Windows 系统
- CMake
- MinGW 或其他支持 Windows 的 g++ 编译器

## 步骤

1. 安装 CMake 和 MinGW。确保 cmake 和 g++ 已正确配置到系统路径。
2. 使用以下命令编译并运行项目:

```shell
.\run.bat
```

## 项目结构

```css
GraphicsLibrary/
|── README.md
├── CMakeLists.txt
├── run.bat
├── src/
│   ├── main.cpp
│   ├── graphics.cpp
│   └── graphics.h
```

\end{lstlisting}

\subsection{CMakeLists.txt}
\begin{lstlisting}[language=CMake]
cmake_minimum_required(VERSION 3.10)

# 项目名称
project(GraphicsLibrary)

# 设置 C++ 标准
set(CMAKE_CXX_STANDARD 11)

# 设置编译器标志以支持 Unicode
if(MINGW)
    set(CMAKE_CXX_FLAGS "${CMAKE_CXX_FLAGS} -DUNICODE -D_UNICODE")
else()
    set(CMAKE_CXX_FLAGS "${CMAKE_CXX_FLAGS} /DUNICODE /D_UNICODE")
endif()

# 设置源文件
set(SOURCE_FILES src/main.cpp src/graphics.cpp)

# 创建可执行文件
add_executable(GraphicsLibrary ${SOURCE_FILES})

# 链接 Windows API 库(Gdi32 用于图形)
target_link_libraries(GraphicsLibrary Gdi32)

# 设置目标为 Windows GUI 程序
set_target_properties(GraphicsLibrary PROPERTIES
    WIN32_EXECUTABLE YES  # 这会告诉 CMake 该项目是一个 Windows GUI 程序
)

\end{lstlisting}

\subsection{run.bat}
\begin{lstlisting}[language=Batch]
@echo off
if exist build rd /s /q build
mkdir build
cd build
cmake -G "MinGW Makefiles" ..
mingw32-make
if exist GraphicsLibrary.exe (
    GraphicsLibrary.exe
)

\end{lstlisting}

\subsection{main.cpp}
\begin{lstlisting}[language=C++]
#include <windows.h>
#include "graphics.h"

LRESULT CALLBACK WindowProc(HWND hwnd, UINT uMsg, WPARAM wParam, LPARAM lParam) {
    switch (uMsg) {
        case WM_PAINT: {
            PAINTSTRUCT ps;
            HDC hdc = BeginPaint(hwnd, &ps);

            // 绘制红色的直线
            DrawLine(hdc, 50, 50, 400, 50, RGB(255, 0, 0));
            
            // 绘制绿色的半圆弧
            DrawArc(hdc, 300, 200, 100, 0, 180, RGB(0, 255, 0));
            
            // 绘制蓝色的椭圆弧
            DrawEllipseArc(hdc, 500, 200, 100, 60, 0, 180, RGB(0, 0, 255));
            
            // 绘制黄色的三角形
            POINT points[] = {{150, 150}, {250, 150}, {200, 250}};
            FillPolygon(hdc, points, 3, RGB(255, 255, 0));
            
            // 绘制紫色的名字
            DrawName(hdc, L"YILIU DONG", 150, 350, RGB(255, 0, 255));

            EndPaint(hwnd, &ps);
        } break;

        case WM_DESTROY:
            PostQuitMessage(0);
            break;

        default:
            return DefWindowProc(hwnd, uMsg, wParam, lParam);
    }
    return 0;
}

int APIENTRY WinMain(HINSTANCE hInstance, HINSTANCE hPrevInstance, LPSTR lpCmdLine, int nCmdShow) {
    const wchar_t CLASS_NAME[] = L"GraphicsWindowClass";

    WNDCLASS wc = {};
    wc.lpfnWndProc = WindowProc;
    wc.hInstance = hInstance;
    wc.lpszClassName = CLASS_NAME;

    RegisterClass(&wc);

    HWND hwnd = CreateWindowEx(0, CLASS_NAME, L"Graphics Library", WS_OVERLAPPEDWINDOW,
                               CW_USEDEFAULT, CW_USEDEFAULT, 800, 600,
                               NULL, NULL, hInstance, NULL);

    if (hwnd == NULL) {
        return 0;
    }

    ShowWindow(hwnd, nCmdShow);
    UpdateWindow(hwnd);

    MSG msg = {};
    while (GetMessage(&msg, NULL, 0, 0)) {
        TranslateMessage(&msg);
        DispatchMessage(&msg);
    }

    return 0;
}

\end{lstlisting}

\subsection{graphics.cpp}
\begin{lstlisting}[language=C++]
#include "graphics.h"
#include <cmath>
#include <vector>
#include <algorithm>
#include <wchar.h>

// 绘制直线的函数
void DrawLine(HDC hdc, int x1, int y1, int x2, int y2, COLORREF color) {
    int dx = abs(x2 - x1), dy = abs(y2 - y1);
    int sx = (x1 < x2) ? 1 : -1, sy = (y1 < y2) ? 1 : -1;
    int err = dx - dy;

    while (true) {
        SetPixel(hdc, x1, y1, color);
        if (x1 == x2 && y1 == y2) break;
        int e2 = err * 2;
        if (e2 > -dy) { err -= dy; x1 += sx; }
        if (e2 < dx) { err += dx; y1 += sy; }
    }
}

// 绘制圆弧的函数
void DrawArc(HDC hdc, int cx, int cy, int radius, int startAngle, int endAngle, COLORREF color) {
    float startRad = startAngle * 3.14159f / 180;
    float endRad = endAngle * 3.14159f / 180;
    float step = 1.0f / radius; // 根据半径动态调整步长

    for (float angle = startRad; angle <= endRad; angle += step) {
        int x = cx + static_cast<int>(radius * cos(angle));
        int y = cy + static_cast<int>(radius * sin(angle));
        SetPixel(hdc, x, y, color);
    }
}

// 绘制椭圆弧的函数
void DrawEllipseArc(HDC hdc, int cx, int cy, int a, int b, int startAngle, int endAngle, COLORREF color) {
    float startRad = startAngle * 3.14159f / 180;
    float endRad = endAngle * 3.14159f / 180;
    float step = 1.0f / std::max(a, b);

    for (float angle = startRad; angle <= endRad; angle += step) {
        int x = cx + static_cast<int>(a * cos(angle));
        int y = cy + static_cast<int>(b * sin(angle));
        SetPixel(hdc, x, y, color);
    }
}

// 填充多边形的函数
void FillPolygon(HDC hdc, POINT *points, int n, COLORREF color) {
    LONG yMin = points[0].y, yMax = points[0].y;
    for (int i = 1; i < n; i++) {
        yMin = std::min(yMin, points[i].y);
        yMax = std::max(yMax, points[i].y);
    }

    for (LONG y = yMin; y <= yMax; y++) {
        std::vector<int> intersections;
        for (int i = 0; i < n; i++) {
            int j = (i + 1) % n;
            if ((points[i].y <= y && points[j].y > y) || (points[i].y > y && points[j].y <= y)) {
                int x = points[i].x + (y - points[i].y) * (points[j].x - points[i].x) / (points[j].y - points[i].y);
                intersections.push_back(x);
            }
        }

        std::sort(intersections.begin(), intersections.end());
        for (size_t i = 0; i + 1 < intersections.size(); i += 2) { // 防止越界
            for (int x = intersections[i]; x <= intersections[i + 1]; x++) {
                SetPixel(hdc, x, y, color);
            }
        }
    }
}

// 绘制单个字母 'Y' 的函数
void DrawLetterY(HDC hdc, int x, int y, COLORREF color) {
    DrawLine(hdc, x, y, x + 10, y + 20, color);    // 左上到中间
    DrawLine(hdc, x + 20, y, x + 10, y + 20, color); // 右上到中间
    DrawLine(hdc, x + 10, y + 20, x + 10, y + 40, color);  // 中间到下
}

// 绘制字母 'I' 的函数
void DrawLetterI(HDC hdc, int x, int y, COLORREF color) {
    DrawLine(hdc, x + 10, y, x + 10, y + 40, color);  // 垂直线
    DrawLine(hdc, x, y, x + 20, y, color);            // 顶部横线
    DrawLine(hdc, x, y + 40, x + 20, y + 40, color);  // 底部横线
}

// 绘制字母 'L' 的函数
void DrawLetterL(HDC hdc, int x, int y, COLORREF color) {
    DrawLine(hdc, x, y, x, y + 40, color);           // 垂直线
    DrawLine(hdc, x, y + 40, x + 20, y + 40, color); // 底部横线
}

// 绘制字母 'U' 的函数
void DrawLetterU(HDC hdc, int x, int y, COLORREF color) {
    DrawLine(hdc, x, y, x, y + 40, color);           // 左垂直线
    DrawLine(hdc, x + 20, y, x + 20, y + 40, color); // 右垂直线
    DrawLine(hdc, x, y + 40, x + 20, y + 40, color); // 底部横线
}

// 绘制字母 'D' 的函数
void DrawLetterD(HDC hdc, int x, int y, COLORREF color) {
    DrawLine(hdc, x, y, x, y + 40, color);           // 垂直线
    DrawArc(hdc, x, y + 20, 20, 270, 450, color); // 半圆弧(右边)
}

// 绘制字母 'O' 的函数
void DrawLetterO(HDC hdc, int x, int y, COLORREF color) {
    DrawArc(hdc, x + 20, y + 20, 20, 0, 360, color); // 完整圆形
}

// 绘制字母 'N' 的函数
void DrawLetterN(HDC hdc, int x, int y, COLORREF color) {
    DrawLine(hdc, x, y, x, y + 40, color);          // 左垂直线
    DrawLine(hdc, x + 20, y, x + 20, y + 40, color); // 右垂直线
    DrawLine(hdc, x, y, x + 20, y + 40, color);      // 斜线
}

// 绘制字母 'G' 的函数
void DrawLetterG(HDC hdc, int x, int y, COLORREF color) {
    DrawArc(hdc, x + 20, y + 20, 20, 0, 270, color); // 弧形(3/4圆)
    DrawLine(hdc, x + 20, y + 20, x + 40, y + 20, color); // 底部横线
}

// 绘制名字的函数
void DrawName(HDC hdc, const wchar_t *name, int x, int y, COLORREF color) {
    for (int i = 0; i < wcslen(name); ++i) {
        wchar_t letter = name[i];
        switch (letter) {
            case L'Y': DrawLetterY(hdc, x, y, color); break;
            case L'I': DrawLetterI(hdc, x, y, color); break;
            case L'L': DrawLetterL(hdc, x, y, color); break;
            case L'U': DrawLetterU(hdc, x, y, color); break;
            case L'D': DrawLetterD(hdc, x, y, color); break;
            case L'O': DrawLetterO(hdc, x, y, color); break;
            case L'N': DrawLetterN(hdc, x, y, color); break;
            case L'G': DrawLetterG(hdc, x, y, color); break;
            default: 
                // 如果字母不在case中,直接绘制字母
                SetTextColor(hdc, color);
                SetBkMode(hdc, TRANSPARENT); // 透明背景
                TextOutW(hdc, x, y, &letter, 1); 
                break;
        }
        x += 38; // 设置字母之间的间距
    }
}

\end{lstlisting}

\subsection{graphics.h}
\begin{lstlisting}[language=C++]
#ifndef GRAPHICS_H
#define GRAPHICS_H

#include <windows.h>

// 绘制直线
void DrawLine(HDC hdc, int x1, int y1, int x2, int y2, COLORREF color);

// 绘制圆弧
void DrawArc(HDC hdc, int cx, int cy, int radius, int startAngle, int endAngle, COLORREF color);

// 绘制椭圆弧
void DrawEllipseArc(HDC hdc, int cx, int cy, int a, int b, int startAngle, int endAngle, COLORREF color);

// 填充多边形
void FillPolygon(HDC hdc, POINT *points, int n, COLORREF color);

// 绘制名字
void DrawName(HDC hdc, const wchar_t *name, int x, int y, COLORREF color);

#endif // GRAPHICS_H

\end{lstlisting}


\end{document}
